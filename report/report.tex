\documentclass[10pt,a4paper]{article}

\usepackage{graphicx}
\usepackage[utf8]{inputenc}
\usepackage[T1]{fontenc}  
\usepackage[english]{babel}
\usepackage{verbatim}
\usepackage{amsmath}
\usepackage{enumerate}
\usepackage{latexsym}
\usepackage{pdfpages}
\usepackage{ifthen}
\newboolean{paper}
\setboolean{paper}{false}

\begin{document}

\title{Team Report : ECMM409 - CA3}
\author {Deep Blue}
\date{\today}

\maketitle

\tableofcontents
\newpage

\section*{Introduction}
\addcontentsline{toc}{section}{Introduction}

As part of the ECMM409 team project, we decided to work on one the
GECCO 2007 competion problems : Ant Wars.\\

This game, involving two opponent ants, takes place on a square
toroidal grid of dimension 11x11. Initially, 15 pieces of food are
randomly generated and placed on the game board, the respective
coordinates of the ants are (5, 2) and (5, 8). Playing successively,
the ants are allowed to move one step in every direction including
diagonal movements : (NW, N, NE, E, SE, S, SW, W). If an ant enters a
cell containing a piece of food, a point is added to its score,
however if ant enters an the adverse cell, the opponent is killed and
cannot play anymore. The game lasts at at most 35 moves per
player. The score being only based on the number of pieces of food
eaten, the ants have to collect as much food as possible from their
environment.\\

The aim of this project was to find the best algorithm representating
a winning ant behaviour we could, using a nature inspired approach,
and we decided to use a strongly-typed genetic programming algorithm
to design our ant.\\

First, this short report will analyse and explain the choice of
genetic programming amongst the numerous nature inspired algorithms to
design our artificial ant. Then, the detail of the algorithm will be
presented, followed by an explanation on the framework implementation
using Haskell. Finally, the results of our computations will be
presented.


\begin{center} 
\begin{verbatim}
+-+-+-+-+-+-+-+-+-+-+-+
|F|F| | | | | | | | | |
+-+-+-+-+-+-+-+-+-+-+-+
| | | | | | | | | | | |
+-+-+-+-+-+-+-+-+-+-+-+
| |F| | | | | | | | | |
+-+-+-+-+-+-+-+-+-+-+-+
| | | | |F| | |F| | | |
+-+-+-+-+-+-+-+-+-+-+-+
| | | | | | | | | | | |
+-+-+-+-+-+-+-+-+-+-+-+
|F| |0| | | | | |1| | |
+-+-+-+-+-+-+-+-+-+-+-+
| | | |F| | | | | | | |
+-+-+-+-+-+-+-+-+-+-+-+
| | | | | | | | | | |F|
+-+-+-+-+-+-+-+-+-+-+-+
| | | |F| |F|F| | | | |
+-+-+-+-+-+-+-+-+-+-+-+
| | |F| |F| | | | | | |
+-+-+-+-+-+-+-+-+-+-+-+
| | | | | | |F| |F| | |
+-+-+-+-+-+-+-+-+-+-+-+
\end{verbatim}
\end{center}
A grid generated by our framework

\section{Road Map}

From the onset of the project, a road map was defined, clearly
identifying the important steps that would have to be completed. Of
course, this list of tasks was slightly updated throughout the
project, to take into account some difficulties related to the
implementation or to the unexpected results of some experiments.

\begin{itemize}
\item implement the framework
\item find and implement rule-based ants
\item run experiments on the ants, evaluate the performances with a tournament
\item implement a memory module to the ants and test its influence on the performances
\item design and implement a genetic programming algorithm
\item conduct experiments on the algorithm parameters
\item graphs on the evolution of the best genetic ant, generation after generation
\end{itemize}

\section{Choice of nature inspired algorithm}

The GECCO web-page stipulates that the solution had to be evolved,
thus, implying that genetic programming had to be used. However, in
the context of this project we had to consider every nature inspired
technique available. The next paragraph lists the different nature
inspired techniques we were told and studied this term and why we
think they were not appropriate to solve this project :

\begin{itemize}
\item Cellular automaton : used to model phenomena
\item Neural network : input : grid - output : go forward or not
  problem : huge set of training data needed
\item Genetic & PSO algorithm : for optimisation problems, exploring solution space
\end{itemize}

Only two approaches seemed to be possible. We could have evolved a
population of neural networks represented by a list of coefficients
using a genetic algorithm, each neural network representing an ant
taking a grid as input and calculating wheher or not the ant should go
forward. But a genetic programming algorithm appeared to be more
natural to solve the ant wars problem.

\section{Genetic programming algorithm details}

Genetic programming is merely a genetic algorithm with programms
usually represented in the infix form using a tree structure.\\

We a used strongly-typed approach for our algorithm. In effect, the
trees were generated according to a predifined grammar and contained
two types of data : boolean and integer values. The operators used
were functions manipulating these two types and were inspired some
rule-based ant algorithm we had previously designed.\\





\section{Framework and algorithm implementation}

\section{Results}

\section*{Conclusion}
\addcontentsline{toc}{section}{Conclusion}

\end{document}
