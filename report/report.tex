% Team report for ECMM409 CA 3
\documentclass[a4paper, 11pt]{article}
\usepackage{fullpage}
\title{ECMM409: Team Report}
\begin{document}
\maketitle

\section{Project Description}

The project is an attempt to solve the Ant Wars problem from the GECCO
2007 conference. The goal of the Ant Wars problem is to evolve an ant
which maximizes the amount of food it collects on a square toroidal
grid in a finite number of steps while in the presence of a second
competing ant. The project involves the design and implementation of a
nature-inspired algorithm which successfully solves the problem and
which is as human-competitive as possible.

\section{Research}

The initial research for the project required investigating the design
and implementation issues associated with the problem along with
research into the rule-based strategies and nature-inspired techniques
to be used by the ant.

\subsection{Nature-Inspired Approaches}

Research was carried out into a number of different nature-inspired
approaches which were assessed according to their applicability to the
problem. Techniques which were initially considered for the project
are given below along with some of the reasons for rejecting them:
\begin{description}
\item[Cellular Automata]
\item[Neural Networks]
\item[Neural Computing]
\item[Genetic Algorithms]
\end{description}
It was decided that genetic programming would be the best technique to
use for the problem as it provides the most feasible way to evolve and
select the ant. Further research was then carried out to find the
correct tree structure to use for strongly-typed genetic programming,
which involved determining which functions to use for the nodes and
how the crossover and mutation parameters were to be implemented.

\subsection{Design and Implementation}

The program is written in the Haskell programming language. Using a
functional programming language for the project was appropriate for the
genetic programming techniques used to solve the problem and leads to
a smaller code base and less implementation time.

\section{Development Road Map}

The first stage of the development process involved implementing a
framework in Haskell which could then be used as a basis for
developing the rule-based and genetic programming methods for solving
the problem. This framework consists of modules for representing a
grid, game, and ant.

After the framework was completed, work started on designing and
implementing a number of simple rule-based strategies for the ant and
also on enabling user input into the program in order to test the
strategies against human players. Experiments were than carried out to
determine which ant had the best tournament performance.

The next stage of the development process involved implementing
strongly-typed genetic programming in Haskell.

\section{Description of Algorithm}

\section{Results}

\end{document}